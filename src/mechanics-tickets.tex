\documentclass{article}
\usepackage[utf8]{inputenc}
\usepackage[russian]{babel}
\usepackage{cmap}
\usepackage{amsfonts,amsmath}
\usepackage{geometry}
\geometry{verbose,{{PAPER_SIZE}}paper,tmargin={{TOP_MARGIN}}cm,bmargin={{BOTTOM_MARGIN}}cm,lmargin={{LEFT_MARGIN}}cm,rmargin={{RIGHT_MARGIN}}cm}
\pdfcompresslevel={{PDF_COMPRESS_LEVEL}}
\usepackage{color}
\usepackage{xcolor}

\pagecolor[HTML]{{{PAGE_COLOR}}}
\color[HTML]{{{TEXT_COLOR}}}

\begin{document}

\begin{center}
    \large{Механика -- краткие ответы на билеты экзамена}\\
    \large{составил Дима Евдокимов}\\
    \large{$\alpha$}\\
\end{center}  

\small
\begin{enumerate}
\item Скорость. Ее компоненты по декартовым координатным осям. Вычисление пройденного пути.
\item Ускорение. Его компоненты по декартовым координатным осям. Тангенциальное, нормальное и полное ускорения.
\item Угловая скорость и угловое ускорение. Связь между угловыми и линейными скоростями и ускорениями.
\item Принцип относительности Галилея. Преобразования Галилея.
\item Законы Ньютона. Границы применимости классической механики.
\item Сила тяжести и вес.
\item Силы трения. Сухое и жидкое трение. Силы внутреннего трения.
\item Закон Кулона, сила Лоренца.
\item Работа и мощность. Работа центральных сил и сил однородного силового поля.
\item Кинетическая энергия материальной точки и системы материальных точек.
\item Потенциальная энергия частицы во внешнем поле сил.
\item Полная механическая энергия замкнутой системы материальных точек.
\item Закон сохранения энергии для частицы, движущейся в консервативном поле сил.
\item Энергия гравитационного взаимодействия двух материальных точек.
\item Космические скорости.
\item Потенциальная энергия деформированной пружины.
\item Связь между потенциальной энергией и силой.
\item Потенциальная яма и потенциальный барьер. Условия равновесия механической системы.
\item Потенциальная энергия взаимодействия.
\item Закон сохранения энергии для системы взаимодействующих частиц.
\item Закон сохранения импульса системы взаимодействующих частиц.
\item Центр масс. Движение центра масс системы материальных точек.
\item Упругое и неупругое соударение.
\item Силы инерции.
\item Момент импульса частицы относительно точки и относительно оси.
\item Момент силы частицы относительно точки и относительно оси.
\item Момент импульса твердого тела, вращающегося вокруг неподвижной оси.
\item Момент инерции, теорема Штейнера.
\item Закон сохранения момента импульса системы взаимодействующих точек.
\item Работа, совершаемая внешними силами при вращении твердого тела относительно неподвижной оси.
\item Кинетическая энергия тела, вращающегося вокруг неподвижной оси.
\item Законы динамики твердого тела.
\item Кинетическая энергия твердого тела при плоском движении.
\item Прецессия гироскопа.
\item Теорема о неразрывности струи.
\item Уравнение Бернулли.
\item Преобразования Лоренца.
\item Длина тела в разных системах отсчета.
\item Промежуток времени между событиями в разных инерциальных системах отсчета. Относительность понятия одновременности.
\item Интервал. Его инвариантность.
\item Времениподобные и пространственноподобные интервалы.
\end{enumerate}
\normalsize

\clearpage

  \section{Скорость. Ее компоненты по декартовым координатным осям. Вычисление пройденного пути.}
    \par
      \textit{Как зависит изменение радиуса-вектора от времени? Что такое скорость? Как найти модуль скорости? Как разложить скорость по декартовым координатным осям? Как найти пройденный путь?}\\
    \par
      При движении материальной точки, ее радиус-вектор может изменяться как по величине, так и по направлению.
    \par
      Фиксируем некоторый момент времени $t$ и пронаблюдаем за изменением радиуса-вектора при прохождении малого (элементарного) времени $\Delta t$. Очевидно, что $\Delta s$ и $\Delta \vec r$ зависит от $\Delta t$, но оказывается, что вектор $\frac{\Delta \vec r}{\Delta t}$ при уменьшении $\Delta t$ практически перестает изменяться как по величине, так и по направлению, т.е. стремится к некоторому пределу:
      \begin{equation}
	\vec v=\lim_{\Delta t\rightarrow 0}{\frac{\Delta \vec r}{\Delta t}}
      \end{equation}
     \par 
      Этот предел называется \textbf{скоростью}.
     \\\\
     \par
      Поскольку $\vec r = f(t)$, можно показать:
      \begin{equation}
	\lim_{\Delta t\rightarrow 0}{\frac{\Delta r}{\Delta t}}=\lim_{\Delta t\rightarrow 0}{\frac{\vec r(t+\Delta t)-\vec r(t)}{\Delta t}}=\vec v = \frac{d\vec r}{dt}=\dot{\vec r}(t)
      \end{equation}
     \\\\
     \par
      Вектор $\frac{\Delta \vec r}{\Delta s}$ лежит на секущей к траектории, но в предельном переходе $\Delta t\rightarrow 0 \Rightarrow \Delta s \rightarrow 0$, поэтому \textbf{скорость направлена по касательной к траектории}
     \par
      Модуль скорости может быть найден так:
      \begin{equation}
	v=|\vec v|=|\lim_{\Delta t\rightarrow 0}{\frac{\Delta \vec r}{\Delta t}}|=\lim_{\Delta t\rightarrow 0}{\frac{|\Delta \vec r|}{\Delta t}}
      \end{equation}
      \par
      Элементарный путь $\Delta s$ отличен от $|\Delta \vec r|$, но в пределе они равны, поэтому:
      \begin{equation}
	v=\lim_{\Delta t\rightarrow 0}{\frac{\Delta s}{\Delta t}}=\frac{ds}{dt}
      \end{equation}
      \par
      Если в пределе взять проекцию вектора $\Delta \vec r$ на некоторую координатную ось, предел будет равен проекции скорости на эту ось, или, что то же самое, первой производной от соответствующей координаты по времени. Таким образом, \textbf{скорость можно разложить по декартовым координантым осям}:
      \begin{equation}
	\vec v = \frac{dx}{dt}\vec e_x+\frac{dy}{dt}\vec e_y+\frac{dz}{dt}\vec e_z = v_x\vec e_x+v_y\vec e_y+v_z\vec e_z
      \end{equation}
      \par
      При малых $\Delta t \; v=\frac{\Delta s}{\Delta t}$. Поэтому, если выразить путь, пройденный материальной точкой за некоторое время $t_2-t_1$ через сумму элементарных путей $\Delta s_i$, он будет равен:
      \begin{equation}
	S_{t_2-t_1}=\sum^{N}_{i=1}{\Delta s_i}=\sum^{N}_{i=1}{v_i\Delta t_i}
      \end{equation}
      \par
      Совершая предельный переход при $\Delta t_i\rightarrow 0$, получаем определенный интеграл:
      \begin{equation}
	S=\int^{t_2}_{t_1}v(t)dt
      \end{equation}
      \par
      Таким образом, \textbf{путь, пройденный материальной точкой за некоторый промежуток времени можно найти, взяв определенный интеграл от скорости по времени}.
  \clearpage
    
  \section{Ускорение. Его компоненты по декартовым координатным осям. Тангенциальное, нормальное и полное ускорения.}    
    \par
      \textit{Что такое ускорение? Как разложить его по декартовым координатным осям? На какие составляющие раскладывается ускорение? Что такое кривизна? Как выразить модуль полного ускорения через составляющие?}\\    
    \par
      Быстрота изменения вектора скорости характеризуется величиной:
      \begin{equation}
	\vec a=\lim_{\Delta t\rightarrow 0}{\frac{\Delta \vec v}{\Delta t}}=\frac{d\vec v}{d t}
      \end{equation}
    \par
      Эта величина называется \textbf{ускорением}.
    \par
      Поскольку ускорение --- первая производная от скорости по времени, \textbf{взяв определенный интеграл, можно найти скорость в любой момент времени с точностью до некоторой константы -- начальной скорости} $v_{t_1}$:
      \begin{equation}
	\vec v(t)=\vec v_{t_1}+\int^{t_2}_{t_1}{\vec a(t)dt}
      \end{equation}
    \par
      Как и скорость, ускорение может быть разложено по декартовым координатным осям:
      \begin{equation}
	\vec a = \frac{dv_x}{dt}\vec e_x+\frac{dv_y}{dt}\vec e_y+\frac{dv_z}{dt}\vec e_z = \frac{d^2x}{dt^2}\vec e_x+\frac{d^2y}{dt^2}\vec e_y+\frac{d^2z}{dt^2}\vec e_z =a_x\vec e_x+a_y\vec e_y+a_z\vec e_z
      \end{equation}
    \par
      Таким образом, \textbf{проекции ускорения на координатные оси равны вторым производным соответствующих координат по времени}.
    \par
      Скорость можно представить как модуль скорости, умноженный на орт скорости $\vec \tau$. Такой орт называется \textbf{тангенциалью}, он направлен по касательной к траектории. Если подставить в выражение ускорения выражение скорости через модуль и орт, получится следующее:
      \begin{equation}
	\vec a = \frac{d}{dt}(v\vec \tau) = \frac{dv}{dt}\vec\tau + v\frac{d\vec\tau}{dt} =  \dot{v}\vec \tau + v\dot{\vec \tau}
      \end{equation}
    \par
      Это значит, что вектор ускорения можно представить как сумму двух векторов, один из которых коллинеарен тангенциали: $\vec a_\tau || \vec\tau = \frac{dv}{dt}\vec\tau$, а другой ей ортагонален и направлен в сторону поворота тангенциали: $\vec a_n\bot\vec\tau=v\frac{d\vec\tau}{dt}$. Эти составляющие вектора ускорения соответственно называются \textbf{тангенциальным} и \textbf{нормальным} ускорениями.
    \par
      Очевидно, что модуль тангенциального ускорения есть первая производная от модуля скорости по времени ($|\vec a_\tau|=|\dot{v}|$). В то же время, нормальное ускорение зависит от типа траектории, в частности от ее \textbf{кривизны} в той или иной точке.
    \par
      \textbf{Кривизна} --- характеристика плоской кривой, которая по определению равна:
      \begin{equation}
	C=\lim_{\Delta s\rightarrow 0}\frac{\Delta \varphi}{\Delta s}=\frac{d\varphi}{ds}
      \end{equation}
    \par
      Где $\Delta\varphi$ --- угол между касательными в точках кривой, отстоящих друг от друга на $\Delta s$. Таким образом, кривизна -- это \textbf{скорость поворота касательной при движении по кривой}. В пределе (на бесконечно малом участке траектории $ds$) с траекторией сливается некоторая окружность, радиус которой называется \textbf{радиусом кривизны} и находится по формуле $R=\frac{1}{C}=\frac{ds}{d\varphi}$. Центр такой окружности называется центром кривизны.
    \par
      Вычислим $\vec a_n$. Поскольку нормальное ускорение ортогонально тангенциали, его можно выразить через некоторый угол $d\varphi$, на который поворачивается тангенциаль с течением элементарного времени $dt$ и нормального вектора $\vec n$, а именно $\dot{\vec\tau}=\frac{d\varphi}{dt}\vec n$, поэтому:
      \begin{equation}
	\vec a_n=v\dot{\vec\tau}=v\frac{d\varphi}{dt}
      \end{equation}
      В пределе $d\varphi=\frac{ds}{R}$, где $R$ -- радиус кривизны в данной точке. Таким образом:
      \begin{equation}
	\vec a_n=v\frac{ds}{Rdt}\vec n=\frac{v^2}{R}\vec n
      \end{equation}
      \textbf{Нормальное ускорение зависит не только от линейной скорости, но и от радиуса кривизны траектории}.
    \par
      Очевидно, что модуль полного ускорения можно выразить через свои составляющие -- тангенциальное и нормальное ускорения. Поскольку составляющие вектора ортогональны, модуль полного ускорения (по т. Пифагора) равен:
      \begin{equation}
	a=\sqrt{a_\tau^2+a_n^2}=\sqrt{{\frac{dv}{dt}}^2+(\frac{v^2}{R})^2}
      \end{equation}
  \clearpage
    
  \section{Угловая скорость и угловое ускорение. Связь между угловыми и линейными скоростями и ускорениями}
    \par
      \textit{Почему поворот не является вектором? Что такое угловая скорость и угловое ускорение? Куда направлены их векторы? Как выразить модуль линейной скорости и ускорение через угловую скорость и ускорение? Как выразить вектор линейной скорости через угловую скорость?}\\
    \par
      \textbf{Поворот не является вектором}, хотя является направленным отрезком. Поворот не подчиняется правилу сложения векторов. Однако, предельно малые повороты можно считать векторами. Векторная величина
      \begin{equation}
	\vec\omega=\lim_{\Delta t\rightarrow 0}{\frac{d\varphi}{dt}}\vec e_\omega=\frac{d\varphi}{dt}\vec e_\omega
      \end{equation}
    \par
      Называется \textbf{угловой скоростью}. Вектор $\vec\omega$ -- аксиальный, т.е. \textbf{направленный вдоль оси вращения в сторону, определяемую правилом правого винта}.
    \par
      Соответственно, модуль угловой скорости равен:
      \begin{equation}
	\omega=\frac{d\varphi}{dt}
      \end{equation}
    \par
      Равномерное вращение можно охарактеризовать \textbf{периодом} T -- временем, за которое тело совершает поворот на $2\pi$. Поэтому при равномерном движении 
      \begin{equation}
	\omega=\frac{2\pi}{T}=2\pi\nu
      \end{equation}
    \par
      Где $\nu$ -- частота вращения. 
    \par
      Изменение вектора угловой скорости со временем характеризуется величиной, называемой \textbf{угловым ускорением}:
      \begin{equation}
	\vec\beta=\lim_{\Delta t\rightarrow 0}{\frac{\Delta\omega}{\Delta t}}\vec e_\omega=\frac{d\vec\omega}{dt}\vec e_\omega
      \end{equation}
    \par
      Вектор углового ускорения также аксиальный, его модуль равен:
      \begin{equation}
	\beta=\lim_{\Delta t\rightarrow 0}{\frac{\Delta\omega}{\Delta t}}=\frac{d\omega}{dt}
      \end{equation}
    \par
      Отдельные точки вращающегося тела имеют различные линейные скорости, поэтому \textbf{величина линейной скорости материальной точки зависит от расстояния до оси вращения}, это легко видно:
      \begin{equation}
	v=\lim_{\Delta t\rightarrow 0}{\frac{\Delta s}{\Delta t}}=\lim_{\Delta t\rightarrow 0}{R\frac{\Delta\varphi}{\Delta t}}=R\frac{d\varphi}{dt}=R\omega
      \end{equation}
    \par
      Отсюда можно выразить \textbf{нормальное ускорение}:
      \begin{equation}
	\vec a_n=\frac{v^2}{R}\vec n=\omega^2R\vec n
      \end{equation}
    \par
      И \textbf{тангенциальное}:
      \begin{equation}
	\vec a_\tau=|\lim_{\Delta t\rightarrow 0}{\frac{\Delta v}{\Delta t}}|\vec\tau=|\lim_{\Delta t\rightarrow 0}{\frac{\Delta (\omega R)}{\Delta t}}|\vec\tau=R|\lim_{\Delta t\rightarrow 0}{\frac{\Delta \omega}{\Delta t}}|\vec\tau=R\frac{d\omega}{dt}\vec\tau=R\beta\vec\tau
      \end{equation}
    \par
      Поскольку $v=\omega R$, можно показать: $v=\omega r \sin\alpha$, где $\alpha$ -- угол между осью вращения и отрезком $r$, соединяющим ось и материальную точку. Из этого, а также из того, что направление $\vec\omega$ подчиняется правилу правого винта, следует то, что
      \begin{equation}
	\vec v=\vec\omega\times\vec r
      \end{equation}
    \par
      И, в частности, если $r=R$ (радиусу кривизны траектории), очевидно:
      \begin{equation}
	\vec v=\vec\omega\times\vec R
      \end{equation}
  \clearpage
    
  \section{Принцип относительности Галилея. Преобразования Галилея.}
    \par
      \textit{Дать определение преобразованиям Галилея, рассказать, когда они верны. Дать определение принципа относительности Галилея.}\\    
    \par
      Пусть система отсчета $x'y'z'$ движется с некоторой скоростью $v_0$ относительно системы отсчета $xyz$ так, что оси $Ox$ и $Ox'$ совпадают, а оси $Oz, Oz'$ и $Oy, Oy'$ соответственно параллельны.
    \par
      Тогда очевидно, что материальная точка $P(x',y',z')$ в системе $xyz$ имеет координаты:
      \begin{equation}
	x=x'+v_0t \;\;\; y=y' \;\;\; z=z'
      \end{equation}
    \par
      Добавив к этой системе предположение о том, что время в двух системах движется одинаково ($t=t'$), получим систему уравнений:
      \begin{equation}
	x=x'+v_0t \;\;\; y=y' \;\;\; z=z' \;\;\; t=t'
      \end{equation}
    \par
      Эта система называется \textbf{преобразованиями Галилея}. Ее можно продифференцировать по времени и получить таким образом \textbf{правило сложения скоростей}:
      \begin{equation}
	v_x=v_x'+v_0 \;\;\; v_y=v_y' \;\;\; v_z=v_z'
      \end{equation}      
    \par
      Преобразования Галилея справедливы для скоростей много меньших скорости света в вакууме.
    \par
      Из преобразований Галилея следует, что ускорение материальной точки не зависит от выбранной \textbf{инерциальной} системы отсчета, поэтому при переходе от одной инерциальной системы к другой, уравнения динамики не изменяются. Следовательно, \textbf{никакими механическими опытами нельзя определить, находится ли данная система отсчета в движении, или покоится}. Этот принцип называется \textbf{принципом относительности Галилея}.
  \clearpage
    
  \section{Законы Ньютона. Границы применимости классической механики.}
    \par
      \textit{Сформулировать три закона Ньютона, определить классическую, квантовую и релятивистскую механику и границы их применимости.}\\    
    \par    
      \paragraph{Первый закон Ньютона.}
      Всякое тело находится в состоянии покоя или равномерного и прямолинейного движения, пока воздействие со стороны других тел не заставит его изменить это состояние. Системы отсчета, в которых выполняется первый закон Ньютона называются \textbf{инерциальными}, и наоборот.
      \paragraph{Второй закон Ньютона.}
      Ускорение всякого тела прямо пропорционально действующей на него \textbf{силе (мере воздействия на тело)} и обратно пропорционально \textbf{массе (мере инертности тела)}:
      \begin{equation}
	\vec a=k\frac{\vec F}{m} \;\; a=k\frac{F}{m}
      \end{equation}
    \par
      В системе СИ единицы измерения подобраны так, что $k=1$, поэтому:
      \begin{equation}
	m\vec a=\vec F
      \end{equation}
    \par
      Можно перейти к \textbf{импульсу тела} $\vec p=m\vec v$:
      \begin{equation}
	m\frac{d\vec v}{dt}=\frac{d(m\vec v)}{dt}=\frac{d\vec p}{dt}=\vec F
      \end{equation}
    \par    
      \paragraph{Третий закон Ньютона.}
      Всякое действие тел друг на друга носит характер взаимодействия; силы, с которыми действуют друг на друга взаимодействующие тела всегда равны по величине и противоположны по направлению.
      \begin{equation}
	\vec F_{1,2}=-\vec F_{2,1}
      \end{equation}
    \par
      Классическая механика применима к телам по своей массе много превосходящим массу одного атома, и движущимися со скоростями много меньшими скорости света в вакууме.
    \par
      Классическая механика --- предельный случай квантовой и релятивистской механики.
    \par
      Квантовая механика -- механика малых масс (сравнимых с массой одного атома)
    \par
      Релятивистская механика -- механика больших скоростей (сравнимых со скоростью света в вакууме)
  \clearpage

  \section{Сила тяжести и вес.}
    \par
      \textit{Что такое сила тяжести, сила реакции опоры, вес?}\\    
    \par
      Под действием силы притяжения к Земле все тела падают с одинаковым ускорением $g$. Это значит, что в системе отсчета, связанной с Землей на всякое тело действует сила $P=mg$. Эта сила называется \textbf{силой тяжести}.
    \par
      Если тело покоится относительно поверхности Земли, то сила тяжести уравновешивается \textbf{силой реакции опоры} $\vec F_r=-\vec P$.
    \par
      По третьему закону Ньютона тело действует на опору или подвес с силой $\vec G=-\vec F_r=P=mg$, называемой весом тела.
    \par
      Важно, что вес тела равен силе тяжести \textbf{только в случае неподвижности или равномерного движения опоры или подвеса}.
    \par
      В противном случае, если опора или подвес движется с ускорением, можно легко найти, чему равен вес:
      \begin{equation}
	\vec P + \vec F_r = m\vec a
      \end{equation}
      \begin{equation}
	(\vec F_r=-\vec G);\;\;(\vec P=m\vec g)
      \end{equation}
      \begin{equation}
	\vec G=m(\vec g - \vec a)
      \end{equation}
  \clearpage      
    
  \section{Силы трения. Сухое и жидкое трение. Силы внутреннего трения.}
    \par
      \textit{Как классифицируются силы трения? Что такое сухое и вязкое трение? Определите силы внутреннего трения.}\\    
    \par    
      Силы трения возникают при перемещении соприкасающихся тел или их частей друг относительно друга.
    \par
      Трение, возникающее при относительном перемещении двух соприкасающихся тел называется \textbf{внешним}.
    \par
      Трение между частями сплошного тела (например, жидкости или газа) называется \textbf{внутренним}.
    \par
      Трение между твердым телом и жидкостью/газом \textbf{считается внутренним}.
    \par
      Трение между поверхностями твердых тел без прослойки называется сухим.
    \par
      Трение между твердым телом и жидкостью/газом называется \textbf{вязким}.
    \par
      Из сухого трения выделяют \textbf{трение скольжения} и \textbf{трение качения}.
    \paragraph{Закон сухого трения.} Максимальная сила трения покоя а также сила трения скольжения не зависит от величины поверхности соприкосновения трущихся тел и пропорциональна силе нормального давления $F_n$:
    \begin{equation}
      F=kF_n
    \end{equation}
    \par
      Здесь $k$ -- коэффицент трения скольжения.
    \par
      Трение качения также зависит от радиуса:
    \begin{equation}
      F=\frac{f}{R}F_n
    \end{equation}
    \par
      Здесь $f$ -- коэффицент трения качения.
    \par
      Вязкое трение в случае трения твердого тела и вязкой среды \textbf{при малых скоростях прямо пропорционально скорости}:
      \begin{equation}
	\vec F = -k_1\vec v
      \end{equation}
    \par
      При \textbf{больших скоростях -- квадрату скорости}:
      \begin{equation}
	\vec F = -k_2v^2\vec e_v
      \end{equation}
    \par
      Важно, что $k_1\neq k_2$, и эти коэффиценты зависят от размеров и формы тела.
  \clearpage
    
  \section{Закон Кулона, сила Лоренца.}
    \par
      \textit{Сформулируйте закон Кулона, определите силу Лоренца.}\\
    \paragraph{Закон Кулона.} Сила взаимодействия двух точечных зарядов в вакууме направлена вдоль прямой, соединяющей эти заряды, пропорциональна их величинам и обратно пропорциональна квадрату расстояния между ними. Она является силой притяжения, если знаки зарядов разные, и силой отталкивания, если эти знаки одинаковы.
    \begin{equation}
      \vec F=k\frac{q_1q_2}{r^2}\vec e_F
    \end{equation}
    \par
      На движущуюся заряженную частицу, находящуюся в электромагнитном поле действует \textbf{сила Лоренца}:
    \begin{equation}
      \vec F=q(\vec E+[\vec v\times \vec B])
    \end{equation}
    \par
      Здесь $q$ -- величина заряда частицы, $\vec v$ -- ее скорость, $\vec E$ -- напряженность электрического поля, $\vec B$ -- магнитная индукция.
  \clearpage    

  \section{Работа и мощность. Работа центральных сил и сил однородного силового поля.}
    \par
      \textit{Что такое работа? Чему равна работа по прямолинейной, криволинейной траектории? Что такое мощность? Что такое центральные силы? Чему равна их работа? Что такое однородное силовое поле?}\\
    \par
      При прямолинейном движении одной материальной точки и постоянной силе, приложенной к ней считается, что сила совершает \textbf{работу}:
      \begin{equation}
	A=F_ss=Fs\cos(F,s)=\vec F \cdot \vec s
      \end{equation}
    \par
      В случае непостоянной силы или криволинейного движения, говорят об элементарной работе:
      \begin{equation}
	dA=\vec F\cdot d\vec s
      \end{equation}
      В этом случае полная работа равна определенному интегралу:
      \begin{equation}
	A=\int_s{\vec F\cdot d\vec s}
      \end{equation}
    \par
      Величина, показывающая, какая работа совершается в единицу времени называется \textbf{мощностью}.
      \begin{equation}
	W=\lim_{\Delta t\rightarrow 0}{\frac{\Delta A}{\Delta t}}=\frac{dA}{dt}
      \end{equation}
      \begin{equation}
	W=\frac{dA}{dt}=\vec F\frac{ds}{dt}=\vec F \cdot \vec v
      \end{equation}
    \par
      \textbf{Центральными силами} называют силы, линия действия которых при любом положении тела проходит через единую точку, называемую \textbf{центром силы}. Работа центральных сил равна:
      \begin{equation}
	dA=\vec F\cdot d\vec r = F\cos(F,r)dr
      \end{equation}
      \begin{equation}
	A=\int^{\vec r_1}_{\vec r_0}{\vec F(\vec r)\cdot d\vec r}
      \end{equation}
    \par
      \textbf{Однородным силовым полем} называют силовое поле, в каждой точке которого сила постоянна. Следовательно, работа сил такого поля равна:
      \begin{equation}
	A=\int^{\vec r_1}_{\vec r_0}{\vec F\cdot d\vec r}
      \end{equation}
  \clearpage    

  \section{Кинетическая энергия материальной точки и системы материальных точек}
    \par
      \textit{Вывести кинетическую энергию. Связать ее с работой над телом. Сформулировать и доказать теорему о кинетической энергии системы.}\\
    \par      
      Пусть тело $(1)$ (материальная точка) действует на соприкосающееся с ним тело $(2)$ с силой $\vec F$. За время $dt$ точка приложения силы получит перемещение $d\vec s=\vec vdt$, т.е. тело $(1)$ совершает над телом $(2)$ работу $dA=\vec F\cdot d\vec s$, которая совершается за счет убыли кинетической энергии $dA=-dT$, т.е. $dT=-\vec F\cdot \vec vdt$. По третьему закону Ньютона тело $(2)$ действует на тело $(1)$ с силой $\vec F'=-\vec F$, таким образом, $\frac{d\vec v}{dt}m=\vec F';\;\;d\vec v=\frac{1}{m}\vec F'dt$. Умножив это выражение скалярно на $m\vec v$, получим $m\vec vd\vec v=-\vec F\vec vdt$, т.е. $dT=m\vec vd\vec v=d(\frac{mv^2}{2})$. Кинетическая энергия материальной точки равна: $T=\frac{mv^2}{2}$. Работа, совершаемая \textbf{над телом}, равна приращению его кинетической энергии ($A=T_2-T_1$).
    \par
      Сформулируем \textbf{теорему о кинетической энергии системы} (ее еще называют \textbf{теоремой об изменении кинетической энергии}).
    \par
      \textbf{Теорема:} изменение кинетической энергии системы равно работе всех внутренних и внешних сил, действующих на тела системы.
    \par
      \textbf{Доказательство:} рассмотрим систему материальных точек с массами $m_1,m_2,m_3,...m_n$, которые движутся со скоростями $\vec v_1,\vec v_2,\vec v_3,...\vec v_n$ и обладают кинетическими энергиями $T_1,...T_n=\frac{1}{2}m_iv_i^2$; для малого промежутка времени: $dT_i=m_i\vec v_id\vec v_i=m_i\vec v_i\frac{d\vec v_i}{dt}dt$. 
      \begin{equation}
	dT_i=m_i\vec a_i d\vec s_i
      \end{equation}
    \par
      По второму закону Ньютона:
      \begin{equation}
	dT_i=\vec F_id\vec S=dA_i
      \end{equation}
      \begin{equation}
	dT=\sum_i{dA_i}
      \end{equation}  
      \begin{equation}
	T_2-T_1=\sum_i{A_i}
      \end{equation}
  \clearpage   
  
  \section{Потенциальная энергия частицы во внешнем поле сил.}
    \par
      \textit{Что такое поле сил? Как они классифицируются? Чему равна потенциальная энергия частицы в поле сил?}\\
    \par     
      \textbf{Полем сил} называют область пространства, в котором на помещенные туда тела действуют силы, зависящие от положения тел в пространстве (или от времени, в случае нестационарных полей).
    \par
      Если силы поля не зависят от времени, поле называют \textbf{стационарным}. В противном случае поле называют \textbf{нестационарным}.
    \par
      Поле называется \textbf{потенциальным (консервативным)}, если работа сил поля не зависит от способа перемещения тела в данном поле из одного положения в другое, а зависит только от начального и конечного положения тела. 
    \par
      Каждой точке потенциального поля можно сопоставить некоторое значение функции $U(\vec r)$ -- \textbf{потенциальную энергию} -- следующим образом: пусть в некоторой исходной нулевой точке $\vec r=\vec 0$ эта функция равна некоторой константе $U(0)$, а во всех остальных точках потенциального поля, она равна работе, которую совершат над телом силы поля при меремещении тела из данной точки в точку $O$.
    \par
      Очевидно (\textit{см. билет 13}), что при перемещении частицы во внешнем силовом поле силы совершают над ним работу, равную \textbf{убыли потенциальной энергии}.
      \begin{equation}
	U_1-U_2=A_{1,2}
      \end{equation}
  \clearpage  
  
   \section{Полная механическая энергия замкнутой системы материальных точек.}
    \par
      \textit{Что такое замкнутая система мат. точек? Чему равна ее полная энергия?}
    \par
      Тело может обладать как потенциальной, так и кинетической энергией. Сумма этих энергий есть \textbf{полная механическая энергия}.
    \par
      \textbf{Замкнутой системой} называют систему, тела которой не взаимодействуют с внешними телами. Для замкнутой системы тел полная механическая энергия равна сумме потенциальной энергии всей системы, а также сумме кинетических энергий ее составляющих:
      \begin{equation}
	E=U+T=U+\sum_{i=1}^{N}\frac{m_iv_i^2}{2}=const
      \end{equation}
    \par
      \textit{см. билет 20}
  \clearpage
  
   \section{Закон сохранения энергии для частицы, движущейся в консервативном поле сил.}
    \par
      \textit{Рассказать про свойство консервативного поля сил. Вывести значение потенциальной энергии, вывести закон сохранения энергии.}\\
    \par
      В консервативном поле сил сила, действующая на частицу обладает следующим свойством: она зависит только от координат, $\vec F(\vec r)=\vec F(x,y,z)$ и существует скалярная функция $\Pi(x,y,z)$ (не путать с потенциальной энергией!) такая, что:
      \begin{equation}
	F_x(x,y,z)=\frac{\partial\Pi}{\partial x}
      \end{equation}
      \begin{equation}
	F_y(x,y,z)=\frac{\partial\Pi}{\partial y}
      \end{equation}
      \begin{equation}
	F_y(x,y,z)=\frac{\partial\Pi}{\partial z}
      \end{equation}
    \par
      Такая функция называется \textbf{потенциалом} или \textbf{силовой функцией}.
    \par
      Тогда элементарная работа сил поля равна:
      \begin{equation}
	\delta A = \vec F\cdot dr = F_xdx+F_ydy+F_zdz
      \end{equation}
      \begin{equation}
	\delta A = \frac{\partial\Pi}{\partial x}dx + \frac{\partial\Pi}{\partial y}dy + \frac{\partial\Pi}{\partial z}dz
      \end{equation}    
      \begin{equation}
	\delta A=d\Pi
      \end{equation} 
    \par
      А полная:
      \begin{equation}
	A_{1,2}=\int_L\vec F\cdot d\vec r=\int_1^2d\Pi=\Pi_2-\Pi_1
      \end{equation}
      Потенциальная энергия $U$ --- это работа, совершаемая силами поля при перемещении материальной точки в некоторое нулевое положение. Принято, что в нулевой точке силовая функция равна нулю. Это значит, что
      \begin{equation}
	U=0-\Pi_1
      \end{equation}
      По теореме об изменении кинетической энергии:
      \begin{equation}
	T_2-T_1=A_{1,2}=U_1-U_2
      \end{equation}
      \begin{equation}
	T_1+U_1=T_2+U_2
      \end{equation}
      --- \textbf{Закон сохранения энергии}
  \clearpage
  
   \section{Энергия гравитационного взаимодействия двух материальных точек.}
    \par
      \textit{Вывести энергию гравитационного взаимодействия двух материальных точек.}\\
    \par      
      По закону всемирного тяготения:
      \begin{equation}
	\vec F_{1,2} = G\frac{m_1m_2}{r_{1,2}^2}\vec e_r
      \end{equation}
    \par
      Тогда элементарная потенциальная энергия равна:
      \begin{equation}
	dU=\vec F_{1,2}\cdot d\vec r_{1,2}=G\frac{m_1m_2}{r_{1,2}^2}dr
      \end{equation}
      \begin{equation}
	U=\int^2_1 dU = \int^2_1 G\frac{m_1m_2}{r_{1,2}^2}dr = Gm_1m_2(-\frac{1}{r_{1,2}}) + c
      \end{equation}
    \par
      Поскольку потенциальная энергия определяется с точностью до константы,
      \begin{equation}
	U=-\frac{Gm_1m_2}{r_1,2}
      \end{equation}
  \clearpage
  
   \section{Космические скорости}
    \par
      \textit{Определить первую, вторую, третью космические скорости.}\\
    \par     
      \textbf{Первая космическая скорость} необходима для вывода спутника на околоземную орбиту.
      \begin{equation}
	\frac{mv^2}{R}=mg\Rightarrow v_1=\sqrt{gR}\approx 8\text{ км/с}
      \end{equation}
    \par     
      \textbf{Вторая космическая скорость} необходима для вывода тела из сферы земного притяжения.
      \begin{equation}
	\frac{mv^2}{2}=G\frac{mM}{R}\Rightarrow v_2=\sqrt{\frac{2GM}{R}}=v_2\sqrt{2gR}\approx 11\text{ км/с}
      \end{equation}
    \par
      \textbf{Третья космическая скорость} необходима для вывода тела за пределы солнечной системы.
      \begin{equation}
	M\rightarrow M_\text{Солнца}\;\;R\rightarrow R_\text{Солнца}
      \end{equation}
      \begin{equation}
	v_2=\sqrt{\frac{2GM}{R}}\approx42\text{ км/с}
      \end{equation}
    \par
      Но делается поправка на движение Земли относительно Солнца. Земля движется относительно Солнца со скоростью $\approx 30$ км/с, поэтому при запуске в направлении орбитального движения Земли скорость в $42$ км/с достигается при скорости $12$ км/с относительно Земли, а при запуске в противоположном направлении при $72$ км/с. С учетом силы притяжения земли, получается $\approx17$ и $\approx73$ км/с соответственно.
  \clearpage
  
   \section{Потенциальная энергия деформированной пружины.}
    \par
      \textit{Вывести потенциальную энергию деформированной пружины.}\\
    \par   
      По закону Гука:
      \begin{equation}
	F=-kx
      \end{equation}
    \par
      Следовательно:
      \begin{equation}
	U(x)=-A=-\int^x_0F_xdx+U(0)=-\int^x_0(-kx)dx+U(0)=\frac{kx^2}{2}+U(0)
      \end{equation}
  \clearpage
  
   \section{Связь между потенциальной энергией и силой}
    \par
      \textit{Вывести силу через градиент потенциальной энергии}\\
    \par
      \textit{(см. билет 13)}
      \begin{equation}
	F_x(x,y,z)=\frac{\partial \Pi}{\partial x}=-\frac{\partial U}{\partial x}
      \end{equation}
      \begin{equation}
	F_y(x,y,z)=\frac{\partial \Pi}{\partial y}=-\frac{\partial U}{\partial y}
      \end{equation}
      \begin{equation}
	F_z(x,y,z)=\frac{\partial \Pi}{\partial z}=-\frac{\partial U}{\partial z}
      \end{equation}
    \par
      т.е.
      \begin{equation}
	\vec F = -(\frac{\partial U}{\partial x}\vec e_x+\frac{\partial U}{\partial y}\vec e_y+\frac{\partial U}{\partial z}\vec e_z)=-\nabla U
      \end{equation}
  \clearpage
  
   \section{Потенциальная яма и потенциальный барьер. Условия равновесия механической системы.}
    \par
      \textit{Определить потенциальную яму и барьер. Описать виды и условия равновесия механической системы. Рассказать про финитное и инфинитное движение.}\\
    \par
      \textbf{Потенциальная яма} -- область пространства, в которой наблюдается локальный минимум потенциальной энергии.
    \par
      \textbf{Потенциальный барьер} -- противоположное понятие, область пространства, разделяющая две другие области с различными или одинаковыми потенциальными энергиями, в которой наблюдается локальный максимум потенциальной энергии.
    \par
      Система находится в \textbf{неустойчивом равновесии}, когда $\frac{d^2U(x)}{dx^2}<0$
    \par
      Система находится в \textbf{устойчивом равновесии}, когда $\frac{d^2U(x)}{dx^2}>0$
    \par
      Система находится в \textbf{безразличном равновесии}, когда $\frac{d^2U(x)}{dx^2}=0$
    \par
      Равновесие системы с несколькими степенями свободы достигается при равновесии по всем направлениям.
    \par
      \textbf{Финитным движением} называется движение, при котором тело не может удалиться на бесконечность (движение в потенциальной яме)
    \par
      \textbf{Инфинитным движением} -- движение, при котором тело может удалиться сколь угодно далеко
  \clearpage
  
   \section{Потенциальная энергия взаимодействия}
    \par
      \textit{Вывести потенциальную энергию взаимодействия двух, трех, $n$ частиц}\\
    \par
      Рассмотрим систему из двух частиц. Первая частица действует на вторую с силой $\vec F_{12}$, а вторая на первую с силой $\vec F_{21}$. Тогда элементарная внутренняя работа равна:
      \begin{equation}
	dA=\vec F_{12}\cdot d\vec r_1 + \vec F_{21}\cdot d\vec r_2 = \vec F_{12}\cdot d\vec r_1 - \vec F_{12}\cdot d\vec r_2 = -\vec F_{12}\cdot d\vec r_{12}
      \end{equation}
    \par
      Следовательно, элементарная потенциальная энергия взаимодействия двух точек равна:
      \begin{equation}
	dU=-dA=\vec F_{12}\cdot d\vec r_{12}
      \end{equation}
    \par
      Рассмотрим систему из трех частиц, которые действуют друг на друга с силами $\vec F_{12}, \vec F_{13}, \vec F_{21}, \vec F_{23}, \vec F_{31}, \vec F_{32} $
    \par
      Тогда, складывая внутренние работы, получим:
      \begin{equation}
	dU=-\sum_{i\neq k}^3\vec F_{ik}\cdot d\vec r_i
      \end{equation}
    \par
      Но можно рассмотреть элементарные потенциальные энергии взаимодействия пар точек, заменяя радиус-векторы векторами, проведенными между точками (как в первом случае):
      \begin{equation}
	A=-\vec F_{12}\cdot d\vec r_{12}-\vec F_{23}\cdot d\vec r_{23}-\vec F_{13}\cdot d\vec r_{13}=-dU_{12}-dU_{23}-dU_{13}
      \end{equation}
    \par
      Поэтому:
      \begin{equation}
	U=\frac{1}{2}(U_{12}+U_{21}+U_{23}+U_{32}+U_{13}+U_{31})
      \end{equation}
    \par
      Это значит, что в случае $n$ частиц:
      \begin{equation}
	U=\frac{1}{2}\sum_{i,k,i\neq k}^n U_{ik}
      \end{equation}      
  \clearpage
  
   \section{Закон сохранения энергии для системы взаимодействующих частиц.}
    \par
      \textit{Вывести полную энергию системы мат. точек и написать следствия.}\\
    \par
      Рассмотрим систему материальных точек. Пусть они взаимодействуют друг с другом с консервативными силами $\vec F_ik$ (внутренними силами). Пусть кроме внутренних сил есть внешние консервативные $\vec F_i$ и неконсервативные $\vec F_i*$. Складывая уравнения движения для каждого из тел, получаем:
      \begin{equation}
	\sum_{i=1}^nm_i\dot{\vec v_i}=\sum_{i=1}^n\sum_{k=1\neq i}^n\vec F_ik+\sum_{i=1}^n \vec F_i + \sum_{i=1}^n \vec F_i*
      \end{equation}
    \par
      Домножим скалярно на $\vec v_idt=d\vec r$:
      \begin{equation}
	\sum_{i=1}^nm_i\dot{\vec v_i}\cdot v_i dt=\sum_{i=1}^n\sum_{k=1\neq i}^n\vec F_ik\cdot d\vec r+\sum_{i=1}^n \vec F_i\cdot d\vec r + \sum_{i=1}^n \vec F_i*\cdot d\vec r
      \end{equation}
      Получается:
      \begin{equation}
	\sum_{i=1}^n \frac{m_iv_i^2}{2}=-dU_\text{взаимод-я}-dU_\text{в поле}+dA*_\text{неконс. сил}
      \end{equation}
    \par
      Таким образом, полная механическая энергия системы равна:
      \begin{equation}
	E=T+U_\text{взаимод-я}+U_\text{в поле}
      \end{equation}
    \par
      Если внешние \textbf{неконсервативные} силы отсутствуют, то
	\begin{equation}
	  E=T+U_\text{взаимод-я}+U_\text{в поле}=const
	\end{equation}
    \par
      Если \textbf{система замкнута}, то
	\begin{equation}
	  E=T+U_\text{взаимод-я}=const
	\end{equation}
      \textbf{Закон:} полная механическая энергия замкнутой системы частиц, между которыми действуют только консервативные силы остается постоянной.
    \par
      Если в системе есть неконсервативные силы, то их работа равна \textbf{приращению полной энергии}:
      \begin{equation}
	A*_\text{неконс. сил}=E_2-E_1
      \end{equation}
  \clearpage
  
   \section{Закон сохранения импульса системы взаимодействующих частиц.}
    \par
      \textit{Рассмотреть систему мат. точек и вывести закон сохранения импульса.}\\
    \par
      Импульс системы частиц:
      \begin{equation}
	\vec p = \sum_{i=1}^n\vec p_i=\sum_{i=1}^n m_i\vec v_i
      \end{equation}
    \par
      Рассмотрим систему из $n$ взаимодействующих частиц:
      \begin{equation}
	\vec F_{ik} = \text{внутренние силы взаимодействия (и конс. и неконс.)}
      \end{equation}
      \begin{equation}
	\vec F_{i} = \text{внешние силы}
      \end{equation}
    \par
      Общее уравнение движения:
      \begin{equation}
	\dot{\vec p_i}=\sum_{k=1\neq i}^n\vec F_{ik}+\vec F_i
      \end{equation}
    \par
      Складывая уравнения движения, получим:
      \begin{equation}
	\frac{d\vec p}{dt}=\sum_{i=1}^n \vec F_i
      \end{equation}
    \par
      \textbf{Закон:} если система замкнута то $\frac{d\vec p}{dt}=0\Rightarrow \vec p=const$
  \clearpage
  
   \section{Центр масс. Движение центра масс системы материальных точек.}
    \par
      \textit{Сформулировать понятие центра масс, выразить полный импульс системы, перейти к преобразованиям Галилея.}\\
    \par      
      \textbf{Центр масс (инерции)} -- это точка $C$ такая, что
      \begin{equation}
	\vec r_c=\frac{m_1\vec r_1+m_2\vec r_2+...+m_n\vec r_n}{m_1+m_2+...+m_n}=\frac{\sum m_i\vec r_i}{\sum m_i}=\frac{\sum m_i\vec r_i}{M}
      \end{equation}
    \par
      Отсюда скорость центра масс:
      \begin{equation}
	\vec v_c=\frac{d\vec r_c}{dt}=\frac{1}{M}(m_1\frac{d\vec r_1}{dt}+m_2\frac{d\vec r_2}{dt}+...+m_n\frac{d\vec r_n}{dt})=\frac{\sum m_i\vec v_i}{M}=\frac{\sum \vec p_i}{M}
      \end{equation}
    \par
      Тогда полный импульс системы равен:
      \begin{equation}
	\vec p=\sum\vec p_i=M\vec v_c
      \end{equation}
    \par
      \textbf{Полный импульс системы равен массе всей системы помноженной на скорость ее центра масс}. Отсюда следует уравнение движения центра масс:
      \begin{equation}
	M\frac{d\vec v_c}{dt}=\sum_{i}\vec F_i
      \end{equation}
    \par
      Если система замкнута то $\vec v_c=const$, различной, в различных системах отсчета
    \par
      \textbf{Система центра масс (Ц-система)} -- это такая инерциальная система отсчета, в которой скорость центра масс равна нулю $\vec v_c=0$.
    \par  
      \textbf{Лабораторная система (Л-система)} -- это такая инерциальная система отсчета, которая связана с наблюдателем.
      Чтобы перейти от Л-системы к Ц-системе, следует найти скорость центра масс по формуле $\vec v_c=\frac{\sum m_i\vec v_i}{M}$ и воспользоваться преобразованиями Галилея (\textit{см. билет 4}).
    \par
      Следует помнить, что в Ц-системе центр масс покоится, а полный импульс системы равен 0.
  \clearpage
  
   \section{Упругое и неупругое соударение.}
    \par
      \textit{Рассказать про упругое и неупругое соударение, рассказать про потери энергии при неупругом соударении, рассказать про ЗСИ и ЗСЭ.}\\
    \par 
      При \textbf{абсолютно неупругом} соударении кинетическая энергия теряется по-максимуму. После удара частицы движутся с одной скоростью (как одно целое), либо покоятся. Закон сохранения энергии не выполняется. Закон сохранения импульса в Л-системе:
      \begin{equation}
	m_1\vec v_1+m_2\vec v_2=(m_1+m_2)\vec v'=(m_1+m_2)\vec v_c
      \end{equation}
    \par
      Потери механической энергии при абсолютно неупругом ударе:
      \begin{equation}
	-\Delta E = -\Delta T = T-T'=\frac{m_1v_1^2}{2}+{m_2v_2^2}{2}-\frac{(m_1+m_2)v'^2}{2}=\frac{m_1m_2}{2(m_1+m_2)}(\vec v_1-\vec v_2)^2
      \end{equation}
    \par
      При \textbf{абсолютно упругом} ударе механическая энергия не переходит в другие виды энергии, потому что тела не деформируются. Следовательно, кинетическая энергия системы сохраняется. По закону сохранения импульса:
      \begin{equation}
	m_1\vec v_1+m_2\vec v_2=m_1\vec v_1'+m_2\vec v_2'
      \end{equation}
    \par
      Полная кинетическая энергия системы до удара и после удара постоянна:
      \begin{equation}
	\frac{m_1v_1^2}{2}+\frac{m_2v_2^2}{2}=\frac{m_1v_1'^2}{2}+\frac{m_2v_2'^2}{2}
      \end{equation}
  \clearpage
  
   \section{Силы инерции.}
    \par
      \textit{Рассказать, откуда берутся силы инерции.}\\
    \par 
      Неинерциальные системы отсчета движутся относительно инерциальных с некоторым ускорением $\vec a$, следовательно, тела в неинерциальных системах отсчета движутся с некоторым ускорением относительно тел в инерциальных системах отсчета.
    \par
      В инерциальных системах отсчета выполняются законы Ньютона. Чтобы они выполнялись и в неинерциальных системах, следует рассмотреть некую силу, которая и дает вышеописанное ускорение телам в этой системе. Такая сила называется \textbf{силой инерции}:
      \begin{equation}
	\vec F_{in}=-m(\vec a_\text{отн. инерц.}-\vec a'_\text{отн. неинерц.})
      \end{equation}
  \clearpage
    
   \section{Момент импульса частицы относительно точки и относительно оси.}
    \par
      \textit{Определить момент импульса относительно точки и относительно оси.}\\
    \par 
      \textbf{Моментом импульса} относительно точки называют векторное произведение:
      \begin{equation}
	\vec M=\vec r\times\vec p
      \end{equation}
    \par
      Момент импульса относительно оси -- это проекция момента импульса относительно точки на оси на эту ось.
      \begin{equation}
	M_z=(\vec r\times\vec p)_\text{Пр. z}
      \end{equation}
    \par
      \textit{Из векторного произведения следуют тривиальные свойства момента импульса, которые легко вывести.}
  \clearpage
    
   \section{Момент силы частицы относительно точки и относительно оси.}
    \par
      \textit{Определить момент силы относительно точки и относительно оси.}\\
    \par 
      \textbf{Моментом силы} относительно точки называют векторное произведение:
      \begin{equation}
	\vec N=\vec r\times \vec F
      \end{equation}
    \par
      Момент силы относительно оси -- это проекция момента силы относительно точки на оси на эту ось.
      \begin{equation}
	N_z=(\vec r\times \vec F)_\text{пр. z}
      \end{equation}
    \par
      \textit{Из векторного произведения следуют тривиальные свойства момента силы, которые легко вывести.}
  \clearpage
    
    \section{Момент импульса твердого тела, вращающегося вокруг неподвижной оси}
    \par
      \textit{Рассказать про плоское, поступательное, вращательное движение и качение. Вывести момент импульса для симметричного тела, вращающегося вокруг неподвижной оси. Перейти к производной момента импульса по времени, перейти ко второму закону Ньютона для вращательного движения.}\\
    \par 
      Если твердое тело движется поступательно, то его движение можно задать одной точкой, т.к. каждая точка тела будет иметь одну и ту же скорость и ускорение.
    \par
      Следует помнить:
      \begin{equation}
	\text{Плоское движение}=\text{Поступательное движение} + \text{Вращательное движение}
      \end{equation}
    \par
      \textbf{Поступательное движение:} у всех точек тела одинаковая скорость, совпадающая со скоростью центра инерции $\vec v_c$
    \par
      \textbf{Вращательное движение:} у всех точек тела одинаковая угловая скорость $\vec \omega$
    \par
      \textbf{Качение:} вращательное + поступательное движение.
    \par
      Найдем момент импульса твердого тела, вращающегося вокруг неподвижной оси. Тело состоит из материальных точек. Переходя к линейным скоростям для каждой такой материальной точки, получаем:
      \begin{equation}
	M_z=\sum M_{iz}=\sum_iR_i^2\omega_z=\omega_z\sum m_i R_i^2
      \end{equation}
    \par
      Обратите внимание, что $R_i\bot z$. Вообще говоря, $\sum m_iR_i^2=I$, поэтому
      \begin{equation}
	M_z=I\omega_z
      \end{equation}
    \par
      Следовательно, в общем случае:
      \begin{equation}
	\vec M=\sum \vec M_i = \sum \vec r_i\times m_i\vec v_i
      \end{equation}
    \par
      А в случае \textbf{осевой симметрии}:
      \begin{equation}
	\vec M=I\vec \omega
      \end{equation}
    \par
      Очевидно, что
      \begin{equation}
	\frac{d\vec M}{dt}=\sum N_\text{внеш.}
      \end{equation}
      \begin{equation}
	\frac{d\vec M_z}{dt}=\sum N_{z_\text{внеш.}}
      \end{equation}
      \begin{equation}
	I\beta_z=\sum N_{z_\text{внеш.}}
      \end{equation}
  \clearpage
    
    \section{Момент инерции, теорема Штейнера.}
    \par
      \textit{Сформулировать понятие момента инерции твердого тела относительно оси, сформулировать теорему Штейнера.}\\
    \par
      \textbf{Момент инерции тела относительно некоторой оси}:
      \begin{equation}
	I=\sum m_i R_i^2
      \end{equation}
    \par
      Следовательно, для однородного тела:
      \begin{equation}
	I=\int R^2dm
      \end{equation}
      \begin{equation}
	I=\int R^2\rho dV
      \end{equation}
    \par
      \textbf{Теорема Штейнера:} момент инерции однородного тела относительно оси, параллельной оси, проходящей через центр инерции, равен сумме момента инерции относительно оси, проходящей через центр инерции и $ma^2$, где $a$ -- расстояние между осями.\\
    \par
  
      \textit{Доказательство теоремы Штейнера настолько тривиально, что автор предпочел выпить чаю вместо его написания.}
  \clearpage
    
    \section{Закон сохранения момента импульса системы взаимодействующих точек.}
    \par
      \textit{Рассмотреть систему мат. точек и вывести закон сохранения момента импульса.}\\
    \par
      (\textit{см. билет 21})
    \par
      Уравнение моментов для $i$-той частицы:
      \begin{equation}
	\frac{d\vec M_i}{dt}=\vec r_i\times \sum_{i\neq k}^n(\vec F_{ik}+\vec F_i)=\sum_{i\neq k}^n (\vec r_i\times\vec F_{ik})+(\vec r_i\times\vec F_i)
      \end{equation}
    \par
      После сложения всех уравнений, получается:
      \begin{equation}
	\frac{d\vec M}{dt}=\sum \vec N_\text{внеш.}
      \end{equation}
    \par
      Отсюда следует, что если момент внешних сил равен 0, \textbf{импульс сохраняется}.
  \clearpage
    
    \section{Работа, совершаемая внешними силами при вращении твердого тела относительно неподвижной оси.}
    \par
      \textit{Рассмотреть элементарную работу внешних сил при вращении через кинетическую энергию вращения.}\\
    \par      
      \begin{equation}
	dA_\text{внеш.}=dT=d(\frac{I\omega^2}{2})=d(\frac{I\omega_z^2}{2})=\frac{I}{2}d(\omega_z^2)=\frac{I}{2}2\omega_zd\omega_z=I\omega_zd\omega_z
      \end{equation}
    \par
      Поскольку $\vec \beta_z=\frac{d\omega_z}{dt}\Rightarrow d\omega_z=\beta_z dt$, 
      \begin{equation}
	dA_\text{внеш.}=I\omega_z\beta_zdt=I\beta_z\omega_zdt=N_zd\varphi
      \end{equation}
      \begin{equation}
	dA_\text{внеш.}=N_zd\varphi
      \end{equation}
      \begin{equation}
	dA_\text{внеш.}=N_\omega d\varphi
      \end{equation}
      \begin{equation}
	dA_\text{внеш.}=\vec Nd\vec\varphi
      \end{equation}
  \clearpage
    
    \section{Кинетическая энергия тела, вращающегося вокруг неподвижной оси.}
    \par
      \textit{Рассмотреть тело из мат. точек и кинетическую энергию точек.}\\
    \par       
      Переходя от угловой скорости к линейным скоростям точек тела, получим:
      \begin{equation}
	T_i=\frac{m_iv_i^2}{2}=\frac{1}{2}\omega^2m_iR_i^2
      \end{equation}
    \par
      Складывая уравнения, получаем:
      \begin{equation}
	T=\sum T_i=\sum\frac{1}{2}\omega^2m_iR_i^2=\frac{1}{2}\omega^2\sum m_iR_i^2
      \end{equation}
    \par
      Поэтому:
      \begin{equation}
	T=\frac{I\omega^2}{2}
      \end{equation}
  \clearpage
    
    \section{Законы динамики твердого тела.}
    \par
      \textit{...}\\
    \par     
      Законы динамики твердого тела:
      \begin{equation}
	m\vec a_c=\sum\vec F_\text{внеш}
      \end{equation}
      \begin{equation}
	\frac{d\vec M}{dt}=\sum\vec N_\text{внеш.}
      \end{equation}
  \clearpage
    
    \section{Кинетическая энергия твердого тела при плоском движении.}
    \par
      \textit{Вывести кинетическую энергию тела в плоском движении, переходя от угловой скорости к линейным скоростям точек.}\\
    \par 
      При плоском движении скорость $i$-й точки:
      \begin{equation}
	\vec v=\vec v_0+\vec\omega\times\vec r_i
      \end{equation}
    \par
      Отсюда ее кинетическая энергия (относительно точки $O$, выбранной произвольно):
      \begin{equation}
	T_i=\frac{m_i\vec v_i^2}{2}=\frac{m_i}{2}(\vec v_0+\vec\omega\times r_i)^2=\frac{m_iv_0^2}{2}+m_i\vec r_i(\vec v_0\times \vec\omega)+\frac{m_i\omega^2R_i^2}{2}
      \end{equation}
    \par
      Складывая уравнения, получаем:
      \begin{equation}
	T=\sum T_i=\frac{mv_0^2}{2}+(\vec v_0\times\omega)m\vec r_c+\frac{I\omega^2}{2}
      \end{equation}
    \par
      Здесь $\vec r_c$ проведен из точки $O$ в центр масс. Если в качестве точки выбрать центр масс, выходит, что
      \begin{equation}
	T=\frac{mv_c^2}{2}+\frac{I_c\omega^2}{2}
      \end{equation}
  \clearpage
    
    \section{Прецессия гироскопа.}
    \par
      \textit{Что такое гироскоп? Что такое прецессия? Связать момент силы, угловую скорость прецессии и момент импульса. Вывести угловую скорость прецессии.}\\
    \par 
      \textbf{Гироскопом} называют массивное симметричное тело, вращающееся с большой скоростью вокруг оси симметрии.
    \par
      Если на гироскоп действуют внешние силы (например, сила тяжести), наблюдается вращение оси вращения гироскопа относительно другой оси, проходящей через точку опоры/подвеса гироскопа. Это явление называется \textbf{прецессией}.
    \par
      Если на гироскоп действуют внешние силы, то момент этих сил становится причиной изменения момента импульса:
      \begin{equation}
	\frac{d\vec M}{dt}=\vec N\Rightarrow d\vec M=\vec Ndt
      \end{equation}
    \par
      Найдем величину $|d\vec M|$:
      \begin{equation}
	|d\vec M|=M\sin\alpha d\varphi = M\sin\alpha \omega' dt
      \end{equation}
      ($\omega'$ -- угловая скорость прецессии) Отсюда:
      \begin{equation}
	d\vec M=(\vec\omega'\times \vec M)dt
      \end{equation}
      \begin{equation}
	d\vec N=(\vec\omega'\times\vec M)
      \end{equation}   
    \par
      Найдем угловую скорость прецессии ($\vec F=m\vec g$) $\omega'$:
      \begin{equation}
	N=mgl\sin\alpha
      \end{equation}
    \par
      Здесь $l$ -- расстояние от точки опоры/подвеса до центра масс гироскопа.
      \begin{equation}
	mgl\sin\alpha=\omega'M\sin\alpha
      \end{equation}
    \par
      Получаем угловую скорость прецессии:
      \begin{equation}
	\omega'=\frac{mgl}{M}=\frac{mgl}{I\omega}
      \end{equation}
  \clearpage
    
    \section{Теорема о неразрывности струи.}
    \par
      \textit{Вывести теорему о неразрывности струи.}\\
    \par       
      Возьмем перпендикулярное к направлению скорости жидкости сечение трубки тока $S$. Предположим, что скорость движения частиц жидкости одинакова во всех точках этого сечения. За время $\Delta t$ через сечение $S$ пройдут все частицы, расстояние которых от $S$ в начальный момент не превышает $v\Delta t$. Следовательно, за время $\Delta t$ через сечение $S$ пройдет объем жидкости, равный $Sv\Delta t$, а за единицу времени через сечени $S$ пройдет объем жидкости равный $Sv$. При одинаковых скоростях, выполняется:
      \begin{equation}
	S_1v_1=S_2v_2
      \end{equation}
    \par
      Отсюда:
      \begin{equation}
	Sv=const
      \end{equation}
      Это положение называется \textbf{теоремой о неразрывности струи}.
  \clearpage
    
    \section{Уравнение Бернулли.}
    \par
      \textit{Из закона сохранения энергии выразить работу объема жидкости и приращение его полной энергии, получить уравнение Бернулли и его следствие.}\\
    \par  
      Рассмотрим идеальную жидкость. При прохождении некоторого объема жидкости по трубке тока, весь рассматриваемый объем получает приращение полной энергии, равное разности полных энергий до и после прохождения. Полная энергия объема равна:
      \begin{equation}
	\frac{\rho\Delta Vv^2}{2}+\rho\Delta Vgh
      \end{equation}
    \par
      Таким образом, приращение полной энергии равно:
      \begin{equation}
	\Delta E=(\frac{\rho\Delta V v_2^2}{2}+\rho\Delta Vgh_2)-(\frac{\rho\Delta V v_1^2}{2}+\rho\Delta Vgh_1)
      \end{equation}
    \par
      Поскольку в идеальной жидкости силы трения отсутствуют, приращение полной энергии равно работе сил давления:
      \begin{equation}
	\Delta E = A = p_1S_1\Delta l_1 - p_2S_2\Delta l_2 = (p_1-p_2)\Delta V
      \end{equation}
    \par
      Приравняв выражение энергии к выражению работы и сократив на $\Delta V$, получим \textbf{уравнение Бернулли}:
      \begin{equation}
	\frac{\rho v_1^2}{2}+\rho gh_1+p_1=\frac{\rho v_2^2}{2}+\rho gh_2+p_2
      \end{equation}
    \par
      Из уравнения Бернулли следует, что вдоль любой линии тока для идеальной жидкости выполняется:
      \begin{equation}
	\frac{\rho v^2}{2}+\rho gh+p=const
      \end{equation}
  \clearpage
    
    \section{Преобразования Лоренца.}
    \par
      \textit{Сформулировать преобразования Лоренца.}\\
    \par  
      В релятивистской механике преобразования Галилея (\textit{см. билет 4}) перестают быть верными, т.к. из правила сложения скоростей следует то, что скорость света для разных систем отсчета разная, а это противоречит принципу ее постоянства.
    \par
      Поэтому в релятивистской механике вместо них используют \textbf{преобразования Лоренца} (системы отсчета выбраны аналогично билету 4):
      \begin{equation}
	x=\frac{x'+v_0t'}{\sqrt{1-v_0^2/c^2}}
      \end{equation}
      \begin{equation}
	t=\frac{t'+(v_0/c^2)x'}{\sqrt{1-v_0^2/c^2}}
      \end{equation}
      \begin{equation}
	y=y'\;\;z=z'
      \end{equation}
  \clearpage
    
    \section{Длина тела в разных системах отсчета.}
    \par
      \textit{Рассмотреть длину стержня в разных системах отсчета.}\\
    \par       
      Рассмотрим стержень, расположенный вдоль оси $x$ в системе $K'$ (\textit{см. билет 4, 37}). Его концы имеют координаты $x_1', x_2'$. Следуя преобразованиям Лоренца, можно найти длину стержня в системе отсчета $x'y'z't'$, связав ее с его длиной в системе отсчета $xyzt$:
      \begin{equation}
	x_1'=\frac{x_1-vt}{\sqrt{1-\beta^2}}\;\;x_2'=\frac{x_2-vh}{\sqrt{1-\beta^2}}\;\;\beta=\frac{v}{c}
      \end{equation}
    \par
      Таким образом:
      \begin{equation}
	x_2'-x_1'=\frac{x_2-x_1}{\sqrt{1-\beta^2}}
      \end{equation}
      \begin{equation}
	l=l_0\sqrt{1-\frac{v^2}{c^2}}
      \end{equation}
    \par
      Здесь $l$ -- длина стержня в системе, относительно которой он движется, а $l_0$ -- в системе, относительно которой он покоится.
  \clearpage
    
    \section{Промежуток времени между событиями в разных инерциальных системах отсчета. Относительность понятия одновременности.}
    \par
      \textit{...}\\
    \par      
      Промежуток времени между двумя событиями в релятивистской физике является относительным, т.е. зависит от системы отсчета:
      \begin{equation}
	\tau = \frac{\tau_0}{\sqrt{1-\frac{v^2}{c^2}}}
      \end{equation}
    \par
      Собственное время $t_0$ всегда меньше, чем то же время, измеренное относительно любой другой системы отсчета. Этот эффект называется релятивистским замедлением времени, он следует из постулата о том, что скорость света в различных системах отсчета постоянна.
  \clearpage
  
    \section{Интервал. Его инвариантность.}
    \par
      \textit{...}\\
    \par    
      Интервалом между событиями в релятивистской механике называют выражение:
      \begin{equation}
	\Delta s = \sqrt{c^2\Delta t^2-\Delta x^2-\Delta y^2-\Delta z^2}
      \end{equation}
    \par
      Или:
      \begin{equation}
	\Delta s = \sqrt{c^2\Delta t^2-\Delta l^2}
      \end{equation}
    \par
      Здесь $\Delta l$ -- расстояние между точками обычного пространства, в которых произошли события. Если события происходят с одной и той же частицей, можно перейти к ее скорости:
    \par
      \begin{equation}
	\Delta s = c\Delta t\sqrt{1-v^2/c^2}
      \end{equation}
    \par
      Интервал является инвариантом.
  \clearpage
  
    \section{Времениподобные и пространственноподобные интервалы.}
    \par
      \textit{...}\\
    \par
      Вещественный интервал называется \textbf{времениподобным}.
    \par
      Мнимый интервал называется \textbf{пространственноподобным}.
  \clearpage    
\end{document} 
 
